%% Generated by Sphinx.
\def\sphinxdocclass{jupyterBook}
\documentclass[letterpaper,10pt,english]{jupyterBook}
\ifdefined\pdfpxdimen
   \let\sphinxpxdimen\pdfpxdimen\else\newdimen\sphinxpxdimen
\fi \sphinxpxdimen=.75bp\relax
%% turn off hyperref patch of \index as sphinx.xdy xindy module takes care of
%% suitable \hyperpage mark-up, working around hyperref-xindy incompatibility
\PassOptionsToPackage{hyperindex=false}{hyperref}
%% memoir class requires extra handling
\makeatletter\@ifclassloaded{memoir}
{\ifdefined\memhyperindexfalse\memhyperindexfalse\fi}{}\makeatother

\PassOptionsToPackage{warn}{textcomp}

\catcode`^^^^00a0\active\protected\def^^^^00a0{\leavevmode\nobreak\ }
\usepackage{cmap}
\usepackage{fontspec}
\defaultfontfeatures[\rmfamily,\sffamily,\ttfamily]{}
\usepackage{amsmath,amssymb,amstext}
\usepackage{polyglossia}
\setmainlanguage{english}



\setmainfont{FreeSerif}[
  Extension      = .otf,
  UprightFont    = *,
  ItalicFont     = *Italic,
  BoldFont       = *Bold,
  BoldItalicFont = *BoldItalic
]
\setsansfont{FreeSans}[
  Extension      = .otf,
  UprightFont    = *,
  ItalicFont     = *Oblique,
  BoldFont       = *Bold,
  BoldItalicFont = *BoldOblique,
]
\setmonofont{FreeMono}[
  Extension      = .otf,
  UprightFont    = *,
  ItalicFont     = *Oblique,
  BoldFont       = *Bold,
  BoldItalicFont = *BoldOblique,
]


\usepackage[Bjarne]{fncychap}
\usepackage[,numfigreset=0,mathnumfig]{sphinx}

\fvset{fontsize=\small}
\usepackage{geometry}


% Include hyperref last.
\usepackage{hyperref}
% Fix anchor placement for figures with captions.
\usepackage{hypcap}% it must be loaded after hyperref.
% Set up styles of URL: it should be placed after hyperref.
\urlstyle{same}

\addto\captionsenglish{\renewcommand{\contentsname}{Humanitarian IM}}

\usepackage{sphinxmessages}



        % Start of preamble defined in sphinx-jupyterbook-latex %
         \usepackage[Latin,Greek]{ucharclasses}
        \usepackage{unicode-math}
        % fixing title of the toc
        \addto\captionsenglish{\renewcommand{\contentsname}{Contents}}
        \hypersetup{
            pdfencoding=auto,
            psdextra
        }
        % End of preamble defined in sphinx-jupyterbook-latex %
        

\title{CCCM IM Handbook}
\date{Dec 14, 2021}
\release{}
\author{Global CCCM Cluster}
\newcommand{\sphinxlogo}{\vbox{}}
\renewcommand{\releasename}{}
\makeindex
\begin{document}

\pagestyle{empty}
\sphinxmaketitle
\pagestyle{plain}
\sphinxtableofcontents
\pagestyle{normal}
\phantomsection\label{\detokenize{intro::doc}}


\begin{sphinxadmonition}{warning}{Warning:}
\sphinxAtStartPar
The handbook is in an early draft stage. Most sections are incomplete and structure of the handbook may be subject to change.
\end{sphinxadmonition}

\begin{sphinxadmonition}{note}{Note:}
\sphinxAtStartPar
This handbook is also available in \sphinxcode{\sphinxupquote{pdf}} and \sphinxcode{\sphinxupquote{kindle}} formats. To use the handbook as an offline mobile app, click “Add to home screen” on Chrome and visit the pages you wish to cache/save.
\end{sphinxadmonition}

\sphinxAtStartPar
This handbook was developed with the aim of providing a consolidated guidance document covering the key aspects of Information Management (IM) for Camp Coordination and Camp Management (CCCM). It was created to fill the gaps in guidance specific to information management in CCCM and to document, consolidate and streamline existing IM practices within the sector.%
\begin{footnote}[1]\sphinxAtStartFootnote
Development of the handbook started in late 2021, with the first draft produced in February 2022.
%
\end{footnote}

\sphinxAtStartPar
The primary audience for this handbook are information management staff of all levels, who are involved, or plan to be involved in IM for CCCM in either a programmes or cluster capacity. The handbook is also relevant to non IM personnel especially coordinators, acknowledging the importance of understanding both data literacy and CCCM analytical norms outside of the IM function.

\sphinxAtStartPar
The handbook is presented in three parts, broadly reflecting the differing use\sphinxhyphen{}cases and audiences.
\begin{itemize}
\item {} 
\sphinxAtStartPar
\sphinxstylestrong{Part 1} provides an overview of the key concepts related to humanitarian information management, which can be applied to all sectors and technical areas in a humanitarian respose.

\item {} 
\sphinxAtStartPar
\sphinxstylestrong{Part 2} builds upon the knowledge from part 1, appling it to the role of information management within CCCM programmes.

\item {} 
\sphinxAtStartPar
\sphinxstylestrong{Part 3} focuses on the role of CCCM IM within cluster coordination, with the key responsibilities for each stage of the Humaitarian Programme Cycle (HPC).

\end{itemize}

\sphinxAtStartPar
The handbook is available as a website (viewable on desktop or phone) or can be download as a pdf or kindle format for offline viewing. While viewing the web version, the left margin shows the three parts of the handbook, containing each chapter. Some of the larger chapters are split into sections (denoted by a downward\sphinxhyphen{}facing arrow). The right margin of the screen are for easy navigation through the content ofeach chapter/section.

\sphinxAtStartPar
Humanitarian approaches and tools grow and change over time, which is particularly evident in the field of information management. This handbook aims to be a \sphinxstyleemphasis{living document} whose contents will be continually updated to reflect our growing knowledge and best practices, and evolution in our approaches and tools in both IM and CCCM.

\begin{figure}[htbp]
\centering
\capstart

\noindent\sphinxincludegraphics[width=400\sphinxpxdimen]{{spirits}.jpg}
\caption{Towards evidence\sphinxhyphen{}based desision making}\label{\detokenize{intro:id2}}\end{figure}

\begin{DUlineblock}{0em}
\item[] \sphinxstylestrong{\Large Feedback}
\end{DUlineblock}

\sphinxAtStartPar
If you have any questions, wish to correct any technical or textual mistakes, or wish to suggest improvements to this handbook, plese get in touch with the Global CCCM Cluster Information Management Officers (IMO) Brian Mc Donald \sphinxhyphen{} \sphinxhref{mailto:bmcdonald@iom.int}{bmcdonald@iom.int} or Alisa Ananbeh \sphinxhyphen{} \sphinxhref{mailto:ananbeh@unhcr.org}{ananbeh@unhcr.org}
\begin{quote}

\sphinxAtStartPar
We would like to thank everybody for their support in developing this handbook and hope that you find it accesible and useful

\begin{flushright}
---the CCCM Cluster team
\end{flushright}
\end{quote}

\begin{DUlineblock}{0em}
\item[] \sphinxstylestrong{\large Acknowledgements}
\end{DUlineblock}

\sphinxAtStartPar
The content in this handbook are drawn from three main sources:
\begin{itemize}
\item {} 
\sphinxAtStartPar
The experience, advice and materials from our CCCM colleagues in the field.

\item {} 
\sphinxAtStartPar
Guidance materials developed at the global\sphinxhyphen{}level from inter\sphinxhyphen{}agency platforms such as the Global Information Management Working Group and from other Clusters.

\item {} 
\sphinxAtStartPar
Many excellent information management trainings including: OCHA’s \sphinxhref{https://www.humanitarianresponse.info/en/operations/simulation-training/caim}{Coordinated Assessmet and Information Management training (CAIM)}; and \sphinxhref{https://www.acaps.org/humanitarian-analysis-programme-hap}{ACAPS’s Humanitarian Analysis Program}.

\end{itemize}


\bigskip\hrule\bigskip



\part{Humanitarian IM}


\chapter{Data Literacy}
\label{\detokenize{part1/data literacy:data-literacy}}\label{\detokenize{part1/data literacy::doc}}
\sphinxAtStartPar
The purpose of this chapter is to introduce the reader to the concept of data, what it is, how it used in humanitarian response and its relevance in the role of information management. This chapter forms an important basis for subsequent chapters as it aims to clearly describe key concepts around data to ensure their clear and shared understanding. This shared vocabulary is vital for the collaboration needed at the various stages of the data’s lifecycle. This chapter is primarily aimed at IM’s but is also relevant to any humanitarian involved to any degree in evidence\sphinxhyphen{}based decision making.%
\begin{footnote}[1]\sphinxAtStartFootnote
Much of this chapter is adapted from \sphinxhref{https://schoolofdata.org/courses/}{School of Data} and IFRC’s \sphinxhref{https://preparecenter.org/toolkit/data-playbook-toolkit/}{Data Playbook}
%
\end{footnote}

\sphinxAtStartPar
\sphinxincludegraphics{{idontthinkitmeanswhatyouthinkitmeans}.jpg}


\section{What is data?}
\label{\detokenize{part1/data literacy:what-is-data}}
\sphinxAtStartPar
Data is the physical representation of information in a manner suitable for communication, interpretation, or processing by human beings or by automatic means.%
\begin{footnote}[2]\sphinxAtStartFootnote
From the UNECE \sphinxhref{https://unece.org/info/Statistics/pub/21878}{Terminology on Statistical Metadata}
%
\end{footnote} It can be structured or unstructured, can come in many different forms (human\sphinxhyphen{}readable or machine\sphinxhyphen{}readable) and can come from any number of sources with using any number of methods. While the terms data, information and knowledge are quite often used interchangeable it is helpful to think of information as data integrated into context, and knowledge as a collection of information, processed in a way that provides learning.


\section{What does it look like?}
\label{\detokenize{part1/data literacy:what-does-it-look-like}}
\sphinxAtStartPar
Data is all around us. Look at your desk and pick an item. Describe the attributes of that item. Perhaps you can describe the items colour, its length, its width, its texture, the materials its constructed from or how effective it is for your work. Data is all around us but is usually messy and unstructured. Processing this information into a structure that can provide sense is at the core of IM. This ‘sense\sphinxhyphen{}making’ is often done using different approaches \sphinxhyphen{} an experienced camp manager may decide to walk into a new camp, walk around it observing it, deciding on what actions they need to prioritize. Another may prefer to set up a list of indicators to measure certain needs in the camp. The approaches are different (and quite often complimentary) but the goal and process are to a large extent the same.

\sphinxAtStartPar
To get from messy data to structured that that can be used \sphinxhyphen{} by itself, or more commonly in conjunction with other datasets \sphinxhyphen{} a degree of organizing, tagging or categorizing must take place. If a survey is used, those categories are determined by the questions asked the type of questions and the response options. When setting these categories it is very important that each person involved with the data \sphinxhyphen{} from the person giving the response, the enumerator right up to those whose programmatic decisions it informs \sphinxhyphen{} has a clear and common understanding of what and how a concept is captured in these categories.


\subsection{Formats}
\label{\detokenize{part1/data literacy:formats}}
\sphinxAtStartPar
Valuable humanitarian data can often start out as paper survey responses, hand written notes (ie. distribution details) or as handwritten notes (such as from Focus Group Discussions). To aid the cleaning, processing and management of this data, digitization may be required. Digital data can be stored in a number of the following formats and is closely linked to the tools used to gather and/or store the data:
\begin{itemize}
\item {} 
\sphinxAtStartPar
\sphinxstylestrong{Tabular data:} By far the most common format for humanitarian data, Excel or Comma Separated Value (CSV) files show data as a table where each column represent as variable in your data and each row ideally represents an observation. %
\begin{footnote}[3]\sphinxAtStartFootnote
This form of data presentation is called \sphinxhref{https://vita.had.co.nz/papers/tidy-data.pdf}{Tidy Data} and is considered as an optimal form of representing data to enable data cleaning and analysis.
%
\end{footnote}

\item {} 
\sphinxAtStartPar
\sphinxstylestrong{Relational databases:} Data cant always be represented in a single table. Quite often there is a need to present the data across multiple tables, showing the linkages(relationships) between variables in different table. Relational databases provide an underlying data model for most modern websites and software. An example use for a relational database could be in the recording of trainings, where one table contains rows, each representing a single training while a second table contains the list of participants. The relationships between these two tables could be defined as \sphinxstyleemphasis{each training can contain multiple participants} and \sphinxstyleemphasis{each participant can attend multiple trainings}

\item {} 
\sphinxAtStartPar
\sphinxstylestrong{APIs} To aid the access and transfer of data, it is very common for modern software systems to have an API, in which other websites (or data analysis tools) can request data from the underlying data store. The most common file format for these is called JSON, a semi\sphinxhyphen{}human readable format with the advantage over tabular formats in that is can represent messy semi\sphinxhyphen{}structured data or complex relationships that would otherwise require a database. %
\begin{footnote}[4]\sphinxAtStartFootnote
A simple example o this is to search by a category on \sphinxhref{https://reliefweb.int/updates}{Reliefweb} and clicking “API at the bottom of the page. This link can then be used by Excel which can show the data fields as a table.
%
\end{footnote}

\item {} 
\sphinxAtStartPar
\sphinxstylestrong{Spatial:} Spatial data formats such as .shp, .gpg, .geojson .geotiff or .dem are used to store 2d or 3d spatial data. Most of these formats can display or export to tabular formats.

\end{itemize}

\begin{sphinxShadowBox}
\sphinxstylesidebartitle{note to self}

\sphinxAtStartPar
add an image showing the appearance of three different filetypes
\end{sphinxShadowBox}


\subsection{Sources}
\label{\detokenize{part1/data literacy:sources}}\begin{itemize}
\item {} 
\sphinxAtStartPar
CODs

\item {} 
\sphinxAtStartPar
HDX

\item {} 
\sphinxAtStartPar
Internal systems

\item {} 
\sphinxAtStartPar
Others

\item {} 
\sphinxAtStartPar
Non traditional sources

\end{itemize}


\section{Key data concepts}
\label{\detokenize{part1/data literacy:key-data-concepts}}\begin{itemize}
\item {} 
\sphinxAtStartPar
measures

\item {} 
\sphinxAtStartPar
indicators

\item {} 
\sphinxAtStartPar
scales

\item {} 
\sphinxAtStartPar
standards

\item {} 
\sphinxAtStartPar
primary vs secondary data

\end{itemize}


\section{Methodologies}
\label{\detokenize{part1/data literacy:methodologies}}\begin{itemize}
\item {} 
\sphinxAtStartPar
KI

\item {} 
\sphinxAtStartPar
FGD

\item {} 
\sphinxAtStartPar
Observations

\item {} 
\sphinxAtStartPar
Non traditional sources

\item {} 
\sphinxAtStartPar
Representativeness

\end{itemize}


\subsection{Sampling}
\label{\detokenize{part1/data literacy:sampling}}
\sphinxAtStartPar
…


\section{IM tips}
\label{\detokenize{part1/data literacy:im-tips}}
\sphinxAtStartPar
…


\bigskip\hrule\bigskip



\chapter{What is Information Management?}
\label{\detokenize{part1/what is information management:what-is-information-management}}\label{\detokenize{part1/what is information management::doc}}
\sphinxAtStartPar
A sample note:

\begin{sphinxadmonition}{note}{Note:}
\sphinxAtStartPar
Here is a note
\end{sphinxadmonition}


\chapter{Collection}
\label{\detokenize{part1/what is information management:collection}}
\sphinxAtStartPar
This section discusses:
\begin{itemize}
\item {} 
\sphinxAtStartPar
Secondary data review

\item {} 
\sphinxAtStartPar
Primary data collection

\end{itemize}


\chapter{Processing}
\label{\detokenize{part1/what is information management:processing}}
\sphinxAtStartPar
This section discusses:
\begin{itemize}
\item {} 
\sphinxAtStartPar
Data quality, cleaning, management.

\end{itemize}

\begin{sphinxShadowBox}
\sphinxstylesidebartitle{a test}

\sphinxAtStartPar
just a test to see if this works and to see how far the text goes across the screen
\end{sphinxShadowBox}

\begin{sphinxShadowBox}
\sphinxstylesidebartitle{}

\begin{sphinxadmonition}{note}{Note:}
\sphinxAtStartPar
Here’s my note!
\end{sphinxadmonition}
\end{sphinxShadowBox}


\chapter{Design \& Acquire}
\label{\detokenize{part1/design:design-acquire}}\label{\detokenize{part1/design::doc}}\begin{quote}

\sphinxAtStartPar
Any information that doesn’t not inform or change an decision is worthless.

\begin{flushright}
---Sam L. Savage
\end{flushright}
\end{quote}
\begin{itemize}
\item {} 
\sphinxAtStartPar
many illustrations start with “data collection” but this should not be the first step

\item {} 
\sphinxAtStartPar
secondary data review vs primary data

\item {} 
\sphinxAtStartPar
formulating the questions that need answering.

\item {} 
\sphinxAtStartPar
create a data analysis plan

\item {} 
\sphinxAtStartPar
common data sources

\item {} 
\sphinxAtStartPar
CODs

\item {} 
\sphinxAtStartPar
methods \sphinxhyphen{} KI. FGD

\item {} 
\sphinxAtStartPar
sampling

\end{itemize}


\chapter{Analysis}
\label{\detokenize{part1/analysis:analysis}}\label{\detokenize{part1/analysis::doc}}
\sphinxAtStartPar
…


\section{The Analysis Spectrum}
\label{\detokenize{part1/analysis:the-analysis-spectrum}}
\sphinxAtStartPar
…


\section{Understanding Bias}
\label{\detokenize{part1/analysis:understanding-bias}}

\chapter{Communication}
\label{\detokenize{part1/communication:communication}}\label{\detokenize{part1/communication::doc}}
\begin{sphinxadmonition}{warning}{Warning:}
\sphinxAtStartPar
This chapter is a first draft and may be subject to change.
\end{sphinxadmonition}

\sphinxAtStartPar
Communication may be the final step in the IMs workflow but is by no means the least important. The best analysis, using the best data collection methods and tools are worthless if the communication of these findings are insufficient to inform decisions.


\section{How to write about numbers}
\label{\detokenize{part1/communication:how-to-write-about-numbers}}
\sphinxAtStartPar
When preparing to write up an analysis, it is important to first consider the following:\sphinxstylestrong{Determine your objectives.} Is the intention to inform or update a group on recent activities? Is it to provide insight on a particular topic? Is it to change peoples understanding or decisions on an particular operational issue? Is it to engage with people to gather feedback or to take action?\sphinxstylestrong{Identify your target audience.} What group or groups are you targeting with the above objectives? You will need to tailor the language(non technical experts may not be familiar with technical language), length (shorter messages may be more suitable for general public consumption) and style (different audiences have different lenses in which they will consume and interpret your message).

\sphinxAtStartPar
There are seven basic principles about writing about numbers: %
\begin{footnote}[1]\sphinxAtStartFootnote
Adapted from The Chicago Guide to Writing about Numbers, by Jane E. Miller
%
\end{footnote}
\begin{enumerate}
\sphinxsetlistlabels{\arabic}{enumi}{enumii}{}{.}%
\item {} 
\sphinxAtStartPar
\sphinxstylestrong{Establish the context for your facts.}  Your text should convey the “who, what, when and where” in which to ground your facts. Don’t just assume that the audience has the same contextual understanding.

\item {} 
\sphinxAtStartPar
\sphinxstylestrong{Pick simple, plausible examples.}  Using examples are a good way to transform abstract numbers to more tangible and relatable to the audiences experiences or understanding. An example of this could be used when describing density of the population of Rohingya refugees in Cox’s Bazar, Bangladesh, where comparing the population number and area of the camp can be compared to that of a comparison city familiar to the audience.

\item {} 
\sphinxAtStartPar
\sphinxstylestrong{Select the right tools and media  for the job.} The three basic tools for presenting quantitative information: prose, tables and charts. Choosing the most appropriate tool (or mix of them) and understanding their strengths and weaknesses, is important. Equally important is to use the most appropriate mix of media. Eg. Reports, interactive dashboards, infographics, video, social media, events.

\item {} 
\sphinxAtStartPar
\sphinxstylestrong{Defining your terms (and be careful with jargon).} Unnecessary use of of acronyms and jargon will likely exclude parts of your audience or cause misunderstanding due to unshared understanding of concepts. If acronyms must be used, it is good practice to show them alongside their long form at the point where they first appear.

\item {} 
\sphinxAtStartPar
\sphinxstylestrong{Reporting and interpreting.}  Describing the numbers around an issue should be support by an explanation of “what does that mean” that explains why that number is important or relevant.

\item {} 
\sphinxAtStartPar
\sphinxstylestrong{Specify magnitude and direction of an association.}  Don’t just say “there are more displaced people in camp A than in camp B”, provide a number quantifying \sphinxstyleemphasis{how} different it is. When explaining the relationship between variables it is also important to be clear on the direction of that relationship. For example “IDPs in Camp A had a lower number of food complaints compared to the previous month”.

\item {} 
\sphinxAtStartPar
\sphinxstylestrong{Summarize patterns.} Rather than presenting a big table or graph showing the data and letting the viewer figure things out for themself it is good to summarize and highlight patterns that contribute to the analysis and message.

\end{enumerate}

\begin{sphinxadmonition}{tip}{Tip:}\begin{itemize}
\item {} 
\sphinxAtStartPar
Tell a story

\item {} 
\sphinxAtStartPar
Choose hooks for your audience

\item {} 
\sphinxAtStartPar
Say it visually

\item {} 
\sphinxAtStartPar
Be transparent with the limitations of your analysis

\end{itemize}
\end{sphinxadmonition}


\section{Data Visualization}
\label{\detokenize{part1/communication:data-visualization}}
\sphinxAtStartPar
Communicating with visuals can an effective way to communicate a message, either on its own or alongside accompanying text. Good visuals can help engaging the audience and quite often are a good way to convey complex information in a simpler form.


\subsection{Choosing the right charts}
\label{\detokenize{part1/communication:choosing-the-right-charts}}
\sphinxAtStartPar
When visualising your data, the choice of chart depends on the quantity and type of data you want to represent; the relationships in that data, and ultimately, whether or not the graph clearly communicates your message.%
\begin{footnote}[2]\sphinxAtStartFootnote
Adapted from the FTs \sphinxhref{http://ft-interactive.github.io/visual-vocabulary/}{Visual Vocabulary}. A similar graphics decision tree, based on the type and number of variables, is available at \sphinxhref{https://www.data-to-viz.com/\#area}{Data\sphinxhyphen{}to\sphinxhyphen{}Viz.com}
%
\end{footnote}

\sphinxAtStartPar
The following is pseudo\sphinxhyphen{}decision tree, to support choosing the most appropriate chart type depending on your data and it relationships.


\subsubsection{Deviation}
\label{\detokenize{part1/communication:deviation}}
\sphinxAtStartPar
Emphasize variations (+/\sphinxhyphen{}) from a fixed reference point. Typically the reference point is zero but it can also be a target or a long\sphinxhyphen{}term average. Can also be used to show sentiment (positive/neutral/negative).

\sphinxAtStartPar
\sphinxstylestrong{Examples:} Showing the number of people entering or exiting a site over a period of time. Showing satisfaction with a component in a training. Demographics pyramid in a site, showing population breakdown by age and gender.

\sphinxAtStartPar
Deviation chart examples

\sphinxAtStartPar
\sphinxincludegraphics{{deviation1}.png}\sphinxstylestrong{Diverging bar:} A simple standard bar chart that can handle both negative and positive magnitude values.

\sphinxAtStartPar
\sphinxincludegraphics{{deviation2}.png}\sphinxstylestrong{Diverging bar:} Perfect for presenting survey results which involve sentiment (eg disagree/neutral/agree).

\sphinxAtStartPar
\sphinxincludegraphics{{deviation3}.png}\sphinxstylestrong{Spine:} Splits a single value into two contrasting components (eg male/female).

\sphinxAtStartPar
\sphinxincludegraphics{{deviation4}.png}\sphinxstylestrong{Surplus/deficit filled line:} The shaded area of these charts allows a balance to be shown – either against a
baseline or between two series.


\subsubsection{Correlation}
\label{\detokenize{part1/communication:correlation}}
\sphinxAtStartPar
Show the relationship between two or more variables. Be mindful that, unless you tell them otherwise, many readers
will assume the relationships you show them to be causal (i.e. one causes the other).

\sphinxAtStartPar
\sphinxstylestrong{Examples:} Showing the relationships between areas of origin and current location of displacement.

\sphinxAtStartPar
Correlation chart examples

\sphinxAtStartPar
\sphinxincludegraphics{{corellation1}.png}\sphinxstylestrong{Scatterplot:} The standard way to show the relationship between two continuous variables, each of which has its own axis.

\sphinxAtStartPar
\sphinxincludegraphics{{corellation2}.png}\sphinxstylestrong{Column + line timeline:} A good way of showing the relationship between an amount (columns) and a rate (line).

\sphinxAtStartPar
\sphinxincludegraphics{{corellation3}.png}\sphinxstylestrong{Connected scatterplot:} Usually used to show how the relationship between 2 variables has changed over time.

\sphinxAtStartPar
\sphinxincludegraphics{{corellation4}.png}\sphinxstylestrong{Bubble:} Like a scatterplot, but adds additional detail by sizing the circles according to a third variable.

\sphinxAtStartPar
\sphinxincludegraphics{{corellation5}.png}\sphinxstylestrong{XY heatmap:} A good way of showing the patterns between 2 categories of data, less effective at showing fine differences in amounts.


\subsubsection{Ranking}
\label{\detokenize{part1/communication:ranking}}
\sphinxAtStartPar
Use where an item’s position in an ordered list is more important than its absolute or relative value. Don’t be afraid to highlight the points of interest.

\sphinxAtStartPar
\sphinxstylestrong{Examples:} Comparing indicators of need. Comparing displacement population figures across sites or districts.

\sphinxAtStartPar
Ranking chart examples

\sphinxAtStartPar
\sphinxincludegraphics{{ranking1}.png}\sphinxstylestrong{Histogram:} Standard bar charts display the ranks of values much more easily when sorted into order..

\sphinxAtStartPar
\sphinxincludegraphics{{ranking2}.png}\sphinxstylestrong{Ordered column:} Same as above but more suited to categories of dates or with short labels.

\sphinxAtStartPar
\sphinxincludegraphics{{ranking3}.png}\sphinxstylestrong{Ordered proportional symbol:} Use when there are big variations between values and/or seeing tne differences
between data is not so important..

\sphinxAtStartPar
\sphinxincludegraphics{{ranking4}.png}\sphinxstylestrong{Slope:} Perfect for showing how ranks have changed over time or vary betweencategories.

\sphinxAtStartPar
\sphinxincludegraphics{{ranking5}.png}\sphinxstylestrong{Lollipop:} Lollipops draw more attention to the data value than standard bar/column and can also show rank and
value ef effectively.

\sphinxAtStartPar
\sphinxincludegraphics{{ranking6}.png}\sphinxstylestrong{Bump:} Effective for showing changing rankings across multiple dates. For large datasets,consider grouping lines
using colour.


\subsubsection{Distribution}
\label{\detokenize{part1/communication:distribution}}
\sphinxAtStartPar
Show values in a dataset and how often they occur. The shape (or ‘skew’) of a distribution can be a memorable way of highlighting the lack of uniformity or equality in the data.

\sphinxAtStartPar
\sphinxstylestrong{Examples:}

\sphinxAtStartPar
Distribution chart examples

\sphinxAtStartPar
\sphinxincludegraphics{{distribution1}.png}\sphinxstylestrong{Histogram:} The standard way to show a statistical distribution \sphinxhyphen{} keep the gaps between columns small to highlight the ‘shape’ of the data.

\sphinxAtStartPar
\sphinxincludegraphics{{distribution2}.png}\sphinxstylestrong{Dot plot:} A simple way of showing the change or range (min/max) of data across multiple categories.

\sphinxAtStartPar
\sphinxincludegraphics{{distribution3}.png}\sphinxstylestrong{Box plot:} Summarise multiple distributions by showing the median (centre) and range of the data.

\sphinxAtStartPar
\sphinxincludegraphics{{distribution4}.png}\sphinxstylestrong{Population pyramid:} A standard way for showing the age and sex breakdown of a population distribution;
effectively, back to back histograms.

\sphinxAtStartPar
\sphinxincludegraphics{{distribution5}.png}\sphinxstylestrong{Beeswarm:} Use to emphasise individual points in a distribution. Points can be sized to an additional variable.
Best with medium sized datasets.


\subsubsection{Change over time}
\label{\detokenize{part1/communication:change-over-time}}
\sphinxAtStartPar
Give emphasis to changing trends. These can be short (intra\sphinxhyphen{}day) movements or extended series traversing decades or centuries: Choosing the correct time period is important to provide suitable context for the reader.

\sphinxAtStartPar
\sphinxstylestrong{Examples:}

\sphinxAtStartPar
Change over time chart examples

\sphinxAtStartPar
\sphinxincludegraphics{{changeovertime1}.png}\sphinxstylestrong{Line:} The standard way to show a changing time series. If data are irregular, consider markers to represent
data points.

\sphinxAtStartPar
\sphinxincludegraphics{{changeovertime2}.png}\sphinxstylestrong{Column:} Columns work well for showing change over time \sphinxhyphen{} but usually best with only one series of data at a time.

\sphinxAtStartPar
\sphinxincludegraphics{{changeovertime3}.png}\sphinxstylestrong{Column and timeline:} A good way of showing the relationship over time between an amount (columns) and a rate (line).

\sphinxAtStartPar
\sphinxincludegraphics{{changeovertime4}.png}\sphinxstylestrong{Slope:} Good for showing changing data as long as the data can be simplifed into 2 or 3 points without missing a key part of story.

\sphinxAtStartPar
\sphinxincludegraphics{{changeovertime5}.png}\sphinxstylestrong{Area chart:} Use with care – these are good at showingchanges to total, but seeing change in components can be very difficult.

\sphinxAtStartPar
\sphinxincludegraphics{{changeovertime6}.png}\sphinxstylestrong{Fan chart (projections):} Use to show the uncertainty in future projections \sphinxhyphen{} usually this grows the further forward to projection.

\sphinxAtStartPar
\sphinxincludegraphics{{changeovertime7}.png}\sphinxstylestrong{Connected scatterplot:} A good way of showing changing data for two variables whenever there is a relatively clear pattern of progression.

\sphinxAtStartPar
\sphinxincludegraphics{{changeovertime8}.png}\sphinxstylestrong{Calendar heatmap:} A great way of showing temporal patterns (daily, weekly, monthly) – at the expense of showing precision in quantity.

\sphinxAtStartPar
\sphinxincludegraphics{{changeovertime9}.png}\sphinxstylestrong{Priestley timeline:} Great when date and duration are key elements of the story in the data.

\sphinxAtStartPar
\sphinxincludegraphics{{changeovertime10}.png}\sphinxstylestrong{Circle timeline:} Good for showing discrete values of varying size across multiple categories (eg earthquakes by continent).

\sphinxAtStartPar
\sphinxincludegraphics{{changeovertime11}.png}\sphinxstylestrong{Streamgraph:} A type of area chart; use when seeing changes in proportions over time is more important than individual values.


\subsubsection{Magnitude}
\label{\detokenize{part1/communication:magnitude}}
\sphinxAtStartPar
Show size comparisons. These can be relative (just being able to see larger/bigger) or absolute (need to see fine differences). Usually these show a ‘counted’ number (for example, barrels, dollars or people) rather than a calculated rate or per cent.

\sphinxAtStartPar
\sphinxstylestrong{Examples:}

\sphinxAtStartPar
Magnitude chart examples

\sphinxAtStartPar
\sphinxincludegraphics{{changeovertime11}.png}\sphinxstylestrong{Streamgraph:} A type of area chart; use when seeing changes in proportions over time is more important than individual values.

\sphinxAtStartPar
\sphinxincludegraphics{{magnitude1}.png}\sphinxstylestrong{Bar:} See above. Good when the data are not time series and labels have long category names.

\sphinxAtStartPar
\sphinxincludegraphics{{magnitude2}.png}\sphinxstylestrong{Paired column:} As per standard column but allows for multiple series. Can become tricky to read with more than 2 series.

\sphinxAtStartPar
\sphinxincludegraphics{{magnitude3}.png}\sphinxstylestrong{Marimekko:} A good way of showing the size and proportion of data at the same time – as long as the data are not too complicated.

\sphinxAtStartPar
\sphinxincludegraphics{{magnitude4}.png}\sphinxstylestrong{Isotype (pictogram):} Excellent solution in some instances – use only with whole numbers (do not slice of an arm to represent a decimal).

\sphinxAtStartPar
\sphinxincludegraphics{{magnitude5}.png}**Radar:**A space\sphinxhyphen{}efficient way of showing value of multiple variables– but make sure they are organised in a way  that makes sense to reader.

\sphinxAtStartPar
\sphinxincludegraphics{{magnitude6}.png}\sphinxstylestrong{Parallel coordinates:} A type of area chart; use when seeing changes in proportions over time is more important than individual values.

\sphinxAtStartPar
\sphinxincludegraphics{{magnitude7}.png}\sphinxstylestrong{Grouped symbol:} An alternative to bar/column charts when being able to count data or highlight individual elements is useful.


\subsubsection{Part\sphinxhyphen{}to\sphinxhyphen{}whole}
\label{\detokenize{part1/communication:part-to-whole}}
\sphinxAtStartPar
Show how a single entity can be broken down into its component elements. If the reader’s interest is solely in the size of the components, consider a magnitude\sphinxhyphen{}type chart instead.

\sphinxAtStartPar
\sphinxstylestrong{Examples:}

\sphinxAtStartPar
Part\sphinxhyphen{}to\sphinxhyphen{}whole chart examples

\sphinxAtStartPar
\sphinxincludegraphics{{parttowhole1}.png}\sphinxstylestrong{Stacked column or bar:} A simple way of showing part\sphinxhyphen{}to\sphinxhyphen{}whole relationships but can be difficult to read with more than a few components.

\sphinxAtStartPar
\sphinxincludegraphics{{parttowhole2}.png}**Radar:**Similar to a pie chart – but the centre can be a good way of making space to include more information bout the data (eg total).

\sphinxAtStartPar
\sphinxincludegraphics{{parttowhole3}.png}**Treemap:**Use for hierarchical part\sphinxhyphen{}to\sphinxhyphen{}whole relationships; can be difficult to read when there are many small segments.

\sphinxAtStartPar
\sphinxincludegraphics{{parttowhole4}.png}**Gridplot:**Good for showing \% information, they work best when used on whole numbers and work well in small multiple layout form.

\sphinxAtStartPar
\sphinxincludegraphics{{parttowhole5}.png}**Venn:**Generally only used for schematic representation.

\sphinxAtStartPar
\sphinxincludegraphics{{parttowhole6}.png}**Waterfall:**Can be useful for showing part\sphinxhyphen{}to\sphinxhyphen{}whole relationships where some of the components are negative.


\subsubsection{Spatial}
\label{\detokenize{part1/communication:spatial}}
\sphinxAtStartPar
Aside from locator maps only used when precise locations or geographical patterns in data are more important to the reader than anything else.

\sphinxAtStartPar
\sphinxstylestrong{Examples:}

\sphinxAtStartPar
Spatial chart examples

\sphinxAtStartPar
\sphinxincludegraphics{{spatial1}.png}**Basic choropleth:**The standard approach for putting data on a map – should always be rates rather than totals and use a sensible base geography.

\sphinxAtStartPar
\sphinxincludegraphics{{spatial2}.png}\sphinxstylestrong{Proportional symbol:} Use for totals rather than rates – be wary that small differences in data will be hard to see.

\sphinxAtStartPar
\sphinxincludegraphics{{spatial3}.png}**Flowmap:**For showing unambiguous movement across a map.

\sphinxAtStartPar
\sphinxincludegraphics{{spatial4}.png}\sphinxstylestrong{Contour map:} For showing areas of equal value on a map. Can use deviation colour schemes for showing +/\sphinxhyphen{} values

\sphinxAtStartPar
\sphinxincludegraphics{{spatial5}.png}\sphinxstylestrong{Dot density:} Used to show the location of individual events/locations – make sure to annotate any patterns the reader should see.

\sphinxAtStartPar
\sphinxincludegraphics{{spatial6}.png}**Heatmap:**Can be useful for showing part\sphinxhyphen{}to\sphinxhyphen{}whole relationships where some of the components are negative.


\subsubsection{Flow}
\label{\detokenize{part1/communication:flow}}
\sphinxAtStartPar
Show the reader volumes or intensity of movement between two or more states or conditions. These might be logical sequences or geographical locations.

\sphinxAtStartPar
\sphinxstylestrong{Examples:}

\sphinxAtStartPar
Flow chart examples

\sphinxAtStartPar
\sphinxincludegraphics{{flow1}.png}**Sankey:**Can be useful for showing part\sphinxhyphen{}to\sphinxhyphen{}whole relationships where some of the components are negative.

\sphinxAtStartPar
\sphinxincludegraphics{{flow2}.png}**Waterfall:**Can be useful for showing part\sphinxhyphen{}to\sphinxhyphen{}whole relationships where some of the components are negative.

\sphinxAtStartPar
\sphinxincludegraphics{{flow3}.png}**Network:**Can be useful for showing part\sphinxhyphen{}to\sphinxhyphen{}whole relationships where some of the components are negative.


\subsection{Visual Design Principles}
\label{\detokenize{part1/communication:visual-design-principles}}
\sphinxAtStartPar
Developed by German psychologists, the Gestalt laws describe how we interpret the complex world around us. They explain why a series of flashing lights appear to be moving. And why we read a sentence like this, \sphinxstyleemphasis{notli ket his ort hat}. Understanding these “laws” can be useful in making sure your message is being conveyed effectively.


\subsubsection{Law of Similarity}
\label{\detokenize{part1/communication:law-of-similarity}}
\sphinxAtStartPar
\sphinxincludegraphics{{lawofsimilarity}.png}The human eye tends to perceive similar elements in a design as a complete picture, shape, or group, even if those elements are separated.
Examples of this could be the use of symbols to signify conflict on a map or the use of colour in dots in a scatter plot that are of the same category.


\subsubsection{Law of Prägnanz}
\label{\detokenize{part1/communication:law-of-pragnanz}}
\sphinxAtStartPar
\sphinxincludegraphics{{lawofpragnanz}.png}People will perceive and interpret ambiguous or complex images as the simplest form possible, because it is the interpretation that requires the least cognitive effort of us. Charts should aim to be as complex as necessary and as simple as possible to convey their meaning. \sphinxhref{https://en.wikipedia.org/wiki/Edward\_Tufte}{Edward Tufte} calls this the data\sphinxhyphen{}ink ratio.


\subsubsection{Law of Proximity}
\label{\detokenize{part1/communication:law-of-proximity}}
\sphinxAtStartPar
\sphinxincludegraphics{{lawofproximity}.png}Objects that are near, or proximate to each other, tend to be grouped together.An example of this could be a grouped bar chart where for the funding for each year is grouped by donor.


\subsubsection{Law of Continuity}
\label{\detokenize{part1/communication:law-of-continuity}}
\sphinxAtStartPar
\sphinxincludegraphics{{lawofcontinuity}.png}Elements that are visually connected are perceived as more related than elements with no connection.
This principle is visible when using a line graph to connect point values.


\subsubsection{Law of Common Region}
\label{\detokenize{part1/communication:law-of-common-region}}
\sphinxAtStartPar
\sphinxincludegraphics{{lawofcommonregion}.png}Elements tend to be perceived into groups if they are sharing an area with a clearly defined boundary.
This law is perhaps most commonly used in maps, where administrative boundaries are shown with solid or dashed lines.

\sphinxAtStartPar
When presenting static charts a useful tip is to use \sphinxhref{https://en.wikipedia.org/wiki/Annotation}{annotation} to guide the viewer through the graph, to put the data in context and to highlight the key relevant facts. %
\begin{footnote}[3]\sphinxAtStartFootnote
Data journalism put increasing emphasis on the need for a good annotation layer, as can be seen by this article from the \sphinxhref{https://www.ft.com/content/4743ce96-e4bf-11e7-97e2-916d4fbac0da}{Financial Times}
%
\end{footnote}


\subsection{Use of Colour}
\label{\detokenize{part1/communication:use-of-colour}}
\sphinxAtStartPar
When choosing colours in your charts its important to understand possible local significance that may be associated to a specific colour. For instance, in one country a colour may signify good luck, whereas in a different country, the same colour could be associated with a non\sphinxhyphen{}state armed group.%
\begin{footnote}[4]\sphinxAtStartFootnote
ACAPS have a great guide on the \sphinxhref{https://www.acaps.org/use-colour-data-display}{Use of Colour in Data Display}
%
\end{footnote}

\sphinxAtStartPar
Where possible, special attention should be taken to ensure that chart remain readable when printed in gray scale and that they are colour blind safe, meaning that the chart should not be confusing for people with red\sphinxhyphen{}green colour blindness (an estimated 8\% of men and 0.4\% of women).

\sphinxAtStartPar
Adding to the previous description of the role of color in perception, the use of colour in scales, particularly maps, typically takes one of the following three forms.%
\begin{footnote}[5]\sphinxAtStartFootnote
\sphinxhref{https://colorbrewer2.org/}{Colorbrewer} is a good resource for picking color palettes. \sphinxhref{https://blog.datawrapper.de/which-color-scale-to-use-in-data-vis/}{Datawrapper} have a good blog post describing the use of different colour scales.
%
\end{footnote}
\begin{enumerate}
\sphinxsetlistlabels{\arabic}{enumi}{enumii}{}{.}%
\item {} 
\sphinxAtStartPar
\sphinxincludegraphics{{sequential-scale}.png} \sphinxstylestrong{Continuous(sequential) scales} used to show values going from low to high. Eg. population density per district.

\item {} 
\sphinxAtStartPar
\sphinxincludegraphics{{diverging-scale}.png} \sphinxstylestrong{Diverging scales}  which visualize difference from a norm, such as \sphinxhref{https://storymaps.arcgis.com/stories/a371cdf9462b4dca9051b9f60a3185bc}{this example} showing location in St Vincent that showed both net inward and outward movements of people following the eruption of a volcano.

\item {} 
\sphinxAtStartPar
\sphinxincludegraphics{{qualitative-scale}.png}\sphinxstylestrong{Categorical(qualitative) scales} used to distinguish different (non numeric) objects eg. a map using different shades for different countries.

\end{enumerate}

\sphinxAtStartPar
Two of the most common ways to respresent colour are RGB and CMYK. RGB, commonly used on websites can be shown as a hex number or RGB number. For printed materials where colour accuracy is important, CYMK is typically used. Not all software supports the CMYK colour space, so if color accuracy is important you may want to use an Adobe tool such as Illustrator or In Design to apply finishing touches to print materials.%
\begin{footnote}[6]\sphinxAtStartFootnote
For a detailed explanation of RGB and CMYK and how they differ, see \sphinxhref{https://en.99designs.ch/blog/tips/correct-file-formats-rgb-and-cmyk/}{here}
%
\end{footnote}


\section{Presenting}
\label{\detokenize{part1/communication:presenting}}
\sphinxAtStartPar
Having a great data collection system, doing great analysis and creating effective visuals don’t necessarily lead to informing or changing decision by themselves. An important skill for IMOs that is often overlooked is the importance of verbal communication and presentation skills, be in in an in\sphinxhyphen{}person context such as a Cluster meeting or as is becoming more common, web\sphinxhyphen{}based calls. These meeting offer an important window of opportunity where, if communicated clearly and in a convincing manner a good analysis can meet it’s objectives.

\sphinxAtStartPar
The following video is an example of effective communication, where the speaker shows a clear understanding of his audience(s), succinctly describes the context, the cause, the call for action (giving specific examples) and the urgency and scale.



\sphinxAtStartPar
When presenting slides, consider the following:
\begin{itemize}
\item {} 
\sphinxAtStartPar
\sphinxstylestrong{Only one idea per slide} Having multiple ideas presented will distract your audience and confuse your key message.

\item {} 
\sphinxAtStartPar
\sphinxstylestrong{Explain your point, then show slide.} Your audience can interpret either the visuals on screen or your spoken message. It is very difficult to both at the same time.

\item {} 
\sphinxAtStartPar
\sphinxstylestrong{Speaker is the star, not the slides.} The slides exist to aide the communication of the speaker, not to distract from it.

\item {} 
\sphinxAtStartPar
\sphinxstylestrong{Never read from the slides.} It portrays a lack of preparedness and dilutes the communication rather than complimenting it.

\item {} 
\sphinxAtStartPar
\sphinxstylestrong{Keep your hands free to move.} Not verbal expression can help the audience relate to the message and can help emphasise key messages.

\item {} 
\sphinxAtStartPar
\sphinxstylestrong{Tell a story to drive home your message} Conveying your message through a narrative is a powerful way to introduce your audience with your key points, for them to engage with the topic and to remember it.

\item {} 
\sphinxAtStartPar
\sphinxstylestrong{Use photos and drawings on slides.} Photos can help bring an emotive human element into otherwise abstract messages. Effective visuals can communicate concepts that would be much harder to explain through written or spoken word alone.

\item {} 
\sphinxAtStartPar
\sphinxstylestrong{Face your audience, not your slides.} You are trying to convince, your audience, not the slides.

\item {} 
\sphinxAtStartPar
\sphinxstylestrong{Avoid complexity.}  Unnecessary complexity is a barrier for comprehension and can cause your audience to disengage with the topic.

\item {} 
\sphinxAtStartPar
\sphinxstylestrong{Rehearse, rehearse, rehearse.}

\end{itemize}


\bigskip\hrule\bigskip



\chapter{Data Responsibility}
\label{\detokenize{part1/data responsibility:data-responsibility}}\label{\detokenize{part1/data responsibility::doc}}
\sphinxAtStartPar
This section describes data security; data protection

\begin{sphinxShadowBox}
\sphinxstylesidebartitle{}

\begin{sphinxadmonition}{note}{Note:}
\sphinxAtStartPar
Here’s my note!
\end{sphinxadmonition}
\end{sphinxShadowBox}

\sphinxAtStartPar
\sphinxstylestrong{Outline}
\begin{itemize}
\item {} 
\sphinxAtStartPar
Main resources: IASC Data Responsibility in Humanitarian Action; ICRC Handbook on Data Protection in Humanitarian Action; IOM Data Protection Manual

\item {} 
\sphinxAtStartPar
Explain what is personal information

\item {} 
\sphinxAtStartPar
Examples of types of data in CCCM

\item {} 
\sphinxAtStartPar
Data protection principles

\end{itemize}

\sphinxAtStartPar
IASC Data Responsibility in Humanitarian Action

\sphinxAtStartPar
ICRC Handbook on Data Protection in Humanitarian Action

\sphinxAtStartPar
IOM Data Protection Manual


\chapter{Tools}
\label{\detokenize{part1/tools:tools}}\label{\detokenize{part1/tools::doc}}

\section{test}
\label{\detokenize{part1/tools:test}}
\sphinxAtStartPar
The choice of software used to create CCCM products can be influenced by factors such as personal preference/familiarity, experience \sphinxhyphen{} some IMs may prefer more advanced tools, time constraints and budget. The following list of software is not a prescriptive list, rather a list of tools preferred by the authors


\subsection{Microsoft Excel}
\label{\detokenize{part1/tools:microsoft-excel}}
\sphinxAtStartPar
…


\subsection{QGIS/ArcGIS}
\label{\detokenize{part1/tools:qgis-arcgis}}
\sphinxAtStartPar
…


\subsection{Inkscape/Adobe Illustrator}
\label{\detokenize{part1/tools:inkscape-adobe-illustrator}}
\sphinxAtStartPar
…


\subsection{Microsoft PowerBI}
\label{\detokenize{part1/tools:microsoft-powerbi}}
\sphinxAtStartPar
…


\subsection{Microsoft Forms/Google Forms}
\label{\detokenize{part1/tools:microsoft-forms-google-forms}}

\subsection{Kobo toolbox}
\label{\detokenize{part1/tools:kobo-toolbox}}

\section{Advanced tools}
\label{\detokenize{part1/tools:advanced-tools}}

\part{CCCM Operations IM}


\chapter{The role of IM in CCCM Operations}
\label{\detokenize{part2/operations:the-role-of-im-in-cccm-operations}}\label{\detokenize{part2/operations::doc}}
\sphinxAtStartPar
\sphinxstylestrong{Todo:} create outline for part 3.


\chapter{Products \& Templates}
\label{\detokenize{part2/products:products-templates}}\label{\detokenize{part2/products::doc}}
\sphinxAtStartPar
note: For some products, the operating environment may dictate wherether certain tools are designed at the CCCM partner level or at the Cluster level. For example, in some countries, the CCCM Cluster designed and manages a single system for Complaint and Feedback, whereas in other contexts each CCCM partner design and manage their own system.


\part{CCCM Cluster IM}


\chapter{The role of IM in the Cluster}
\label{\detokenize{part3/cluster:the-role-of-im-in-the-cluster}}\label{\detokenize{part3/cluster::doc}}
\sphinxAtStartPar
\sphinxstylestrong{Todo:} create outline for part 3.


\chapter{The HPC Cycle}
\label{\detokenize{part3/hpccycle:the-hpc-cycle}}\label{\detokenize{part3/hpccycle::doc}}
\sphinxAtStartPar
work in progress..


\section{Preparedness}
\label{\detokenize{part3/preparedness:preparedness}}\label{\detokenize{part3/preparedness::doc}}

\section{Needs Assessment \& Analysis}
\label{\detokenize{part3/needs assessment:needs-assessment-analysis}}\label{\detokenize{part3/needs assessment::doc}}

\section{Strategic Planning}
\label{\detokenize{part3/strategic planning:strategic-planning}}\label{\detokenize{part3/strategic planning::doc}}

\section{Resource Mobilization}
\label{\detokenize{part3/resource mobilization:resource-mobilization}}\label{\detokenize{part3/resource mobilization::doc}}

\section{Implementation \& Monitoring}
\label{\detokenize{part3/implementation and monitoring:implementation-monitoring}}\label{\detokenize{part3/implementation and monitoring::doc}}

\section{Operational Peer Review \& Evaluation}
\label{\detokenize{part3/opr:operational-peer-review-evaluation}}\label{\detokenize{part3/opr::doc}}

\chapter{Products \& Templates}
\label{\detokenize{part3/products:products-templates}}\label{\detokenize{part3/products::doc}}
\sphinxAtStartPar
The following list of CCCM Cluster IM products are a showcase of good examples from various contexts. While they can be reused for other contexts it is important to understand the specific information needs messages and audiences for your context.
\begin{itemize}
\item {} 
\sphinxAtStartPar
\sphinxstylestrong{Operation Presence (3W):}

\item {} 
\sphinxAtStartPar
Contact list, meeting minutes and meeting atterndace sheet.

\item {} 
\sphinxAtStartPar
Site profiles.

\item {} 
\sphinxAtStartPar
Site maps.

\item {} 
\sphinxAtStartPar
Complaints and Feedback Overview

\end{itemize}


\section{Cluster IM checklist}
\label{\detokenize{part3/products:cluster-im-checklist}}
\sphinxAtStartPar
{[} {]} Compile a contact list containing all partners, other cluster focal points, OCHA and donors is in\sphinxhyphen{}place, with an easy way for people to subscribe/unsubscribe (email footer, webpage, mailchimp)
{[} {]} Ensure that the CCCM Cluster webpages on \sphinxhref{http://humanitarianresponse.info}{humanitarianresponse.info} and \sphinxhref{http://cccmcluster.org}{cccmcluster.org} have at minumim, information showing the clusters strategy, current activities, key focal points and key documents.
{[} {]} Have a cluster site master list with clear alignment with cluster partners and data collection agencies.
{[} {]} Develop a 4W spreadsheet for gathering data on what CCCM activities are being conducted in what locations by partners.


\chapter{Conclusion}
\label{\detokenize{conclusion:conclusion}}\label{\detokenize{conclusion::doc}}
\sphinxAtStartPar
…


\section{Further Reading}
\label{\detokenize{conclusion:further-reading}}
\sphinxAtStartPar
…







\renewcommand{\indexname}{Index}
\printindex
\end{document}